% Options for packages loaded elsewhere
\PassOptionsToPackage{unicode}{hyperref}
\PassOptionsToPackage{hyphens}{url}
\PassOptionsToPackage{dvipsnames,svgnames,x11names}{xcolor}
%
\documentclass[
]{interact}

\usepackage{amsmath,amssymb}
\usepackage{iftex}
\ifPDFTeX
  \usepackage[T1]{fontenc}
  \usepackage[utf8]{inputenc}
  \usepackage{textcomp} % provide euro and other symbols
\else % if luatex or xetex
  \usepackage{unicode-math}
  \defaultfontfeatures{Scale=MatchLowercase}
  \defaultfontfeatures[\rmfamily]{Ligatures=TeX,Scale=1}
\fi
\usepackage{lmodern}
\ifPDFTeX\else  
    % xetex/luatex font selection
\fi
% Use upquote if available, for straight quotes in verbatim environments
\IfFileExists{upquote.sty}{\usepackage{upquote}}{}
\IfFileExists{microtype.sty}{% use microtype if available
  \usepackage[]{microtype}
  \UseMicrotypeSet[protrusion]{basicmath} % disable protrusion for tt fonts
}{}
\makeatletter
\@ifundefined{KOMAClassName}{% if non-KOMA class
  \IfFileExists{parskip.sty}{%
    \usepackage{parskip}
  }{% else
    \setlength{\parindent}{0pt}
    \setlength{\parskip}{6pt plus 2pt minus 1pt}}
}{% if KOMA class
  \KOMAoptions{parskip=half}}
\makeatother
\usepackage{xcolor}
\setlength{\emergencystretch}{3em} % prevent overfull lines
\setcounter{secnumdepth}{5}
% Make \paragraph and \subparagraph free-standing
\makeatletter
\ifx\paragraph\undefined\else
  \let\oldparagraph\paragraph
  \renewcommand{\paragraph}{
    \@ifstar
      \xxxParagraphStar
      \xxxParagraphNoStar
  }
  \newcommand{\xxxParagraphStar}[1]{\oldparagraph*{#1}\mbox{}}
  \newcommand{\xxxParagraphNoStar}[1]{\oldparagraph{#1}\mbox{}}
\fi
\ifx\subparagraph\undefined\else
  \let\oldsubparagraph\subparagraph
  \renewcommand{\subparagraph}{
    \@ifstar
      \xxxSubParagraphStar
      \xxxSubParagraphNoStar
  }
  \newcommand{\xxxSubParagraphStar}[1]{\oldsubparagraph*{#1}\mbox{}}
  \newcommand{\xxxSubParagraphNoStar}[1]{\oldsubparagraph{#1}\mbox{}}
\fi
\makeatother


\providecommand{\tightlist}{%
  \setlength{\itemsep}{0pt}\setlength{\parskip}{0pt}}\usepackage{longtable,booktabs,array}
\usepackage{calc} % for calculating minipage widths
% Correct order of tables after \paragraph or \subparagraph
\usepackage{etoolbox}
\makeatletter
\patchcmd\longtable{\par}{\if@noskipsec\mbox{}\fi\par}{}{}
\makeatother
% Allow footnotes in longtable head/foot
\IfFileExists{footnotehyper.sty}{\usepackage{footnotehyper}}{\usepackage{footnote}}
\makesavenoteenv{longtable}
\usepackage{graphicx}
\makeatletter
\newsavebox\pandoc@box
\newcommand*\pandocbounded[1]{% scales image to fit in text height/width
  \sbox\pandoc@box{#1}%
  \Gscale@div\@tempa{\textheight}{\dimexpr\ht\pandoc@box+\dp\pandoc@box\relax}%
  \Gscale@div\@tempb{\linewidth}{\wd\pandoc@box}%
  \ifdim\@tempb\p@<\@tempa\p@\let\@tempa\@tempb\fi% select the smaller of both
  \ifdim\@tempa\p@<\p@\scalebox{\@tempa}{\usebox\pandoc@box}%
  \else\usebox{\pandoc@box}%
  \fi%
}
% Set default figure placement to htbp
\def\fps@figure{htbp}
\makeatother

\usepackage{fancyhdr}
\pagestyle{fancy}
\fancyhead{}
\fancyhead[LL]{}
\fancyhead[R]{}
\fancyfoot[L]{}
\fancyfoot[C]{}
\fancyfoot[R]{}
\usepackage{orcidlink}
\makeatletter
\@ifpackageloaded{caption}{}{\usepackage{caption}}
\AtBeginDocument{%
\ifdefined\contentsname
  \renewcommand*\contentsname{Table of contents}
\else
  \newcommand\contentsname{Table of contents}
\fi
\ifdefined\listfigurename
  \renewcommand*\listfigurename{List of Figures}
\else
  \newcommand\listfigurename{List of Figures}
\fi
\ifdefined\listtablename
  \renewcommand*\listtablename{List of Tables}
\else
  \newcommand\listtablename{List of Tables}
\fi
\ifdefined\figurename
  \renewcommand*\figurename{Figure}
\else
  \newcommand\figurename{Figure}
\fi
\ifdefined\tablename
  \renewcommand*\tablename{Table}
\else
  \newcommand\tablename{Table}
\fi
}
\@ifpackageloaded{float}{}{\usepackage{float}}
\floatstyle{ruled}
\@ifundefined{c@chapter}{\newfloat{codelisting}{h}{lop}}{\newfloat{codelisting}{h}{lop}[chapter]}
\floatname{codelisting}{Listing}
\newcommand*\listoflistings{\listof{codelisting}{List of Listings}}
\makeatother
\makeatletter
\makeatother
\makeatletter
\@ifpackageloaded{caption}{}{\usepackage{caption}}
\@ifpackageloaded{subcaption}{}{\usepackage{subcaption}}
\makeatother

\usepackage{bookmark}

\IfFileExists{xurl.sty}{\usepackage{xurl}}{} % add URL line breaks if available
\urlstyle{same} % disable monospaced font for URLs
\hypersetup{
  colorlinks=true,
  linkcolor={blue},
  filecolor={Maroon},
  citecolor={Blue},
  urlcolor={Blue},
  pdfcreator={LaTeX via pandoc}}


\author{}

\thanks{CONTACT: }
\begin{document}
\captionsetup{labelsep=space}


`No one can craft a jutti like I do'

నేను చేసినట్టు జుత్తీ చేసేవాళ్ళు ఇంకెవరూ లేరు

నేను చేసినంత బాగా జుత్తీ ఇంకెవరూ చేయలేరు

In the village of Rupana, veteran shoemaker Hans Raj is the only artisan
who still makes leather juttis by hand, a craft requiring great skill
and precision, and traditionally practised by Dalit families from Punjab

``Lai de ve jutti mainu, Muktsari kadai wali, Pairan wich mere channa,
jachugi payi bahali'\,'

``Buy me a jutti , the one with Muktsar's embroidery, In my feet, oh my
beloved, it will look amazing.''

Hans Raj tightens his grip on the coarse cotton thread. Using a sharp
steel needle to guide it, the veteran shoemaker pierces the tough
leather, skillfully moving the needle in and out roughly 400 times to
hand-stitch a pair of Punjabi juttis (closed shoes). As he does so, his
heavy sighs followed by `hmms' punctuate the silence.

In the village of Rupana in Sri Muktsar Sahib district of Punjab, Hans
Raj is the only artisan who makes these juttis the traditional way.

``Most people are unaware how a Punjabi jutti is made and who crafts it.
There is a common misconception that machines make it. But from
preparation to stitching, everything is done by hand,'' says the
63-year-old artisan who has been crafting juttis for nearly half a
century now. ``Wherever you go, Muktsar, Malout, Gidderbaha or Patiala,
no one can meticulously craft a jutti like I do,'' Hans Raj says
matter-of-factly.

Every day, starting at 7 a.m., he sits on a cotton mattress on the floor
near the entrance of his rented workshop, part of the walls covered with
a collection of Punjabi juttis for both men and women. A pair is priced
between Rs. 400 to Rs. 1,600, and he says he can earn about Rs. 10,000 a
month from this livelihood.

Left: Hans Raj's rented workshop where he hand stitches and crafts
leather juttis. Right: Inside the workshop, parts of the walls are
covered with juttis he has made

Hansraj has been practicing this craft for nearly half a century. He
rolls the extra thread between his teeth before piercing the tough
leather with the needle

Leaning against the weathered wall, he spends the next 12 hours crafting
handmade shoes. The place where he leans his tired back on the wall is
patchy -- the cement has worn off exposing the bricks underneath. ``The
body aches, especially the legs,'' Hans Raj says, massaging his knee
joints. In summer, he says, ``We get dane je {[}boils{]} on the back
from all the sweating that causes pain.''

Hans Raj learnt the craft when he was around 15, and he was tutored by
his father. ``I was more interested in exploring the outdoors. Some days
I would sit down to learn, some days I would not.'' But as he grew up
and the pressure to work increased, so did the hours of being seated.

Speaking in a mix of Punjabi and Hindi he says, ``this work needs bariki
{[}precision{]}.'' Hans Raj has been working without glasses for years,
``but I have started noticing changes to my eyesight now. If I work for
many hours, my eyes feel the strain. I see two of everything.''

During a regular work day, he drinks tea and listens to the news, songs
and cricket commentary on his radio. His favourite programme is the ``
pharmaishi programme,'' where listeners' request of old Hindi and
Punjabi songs are played. He himself has never called up the radio
station to request any song saying, ``I don't understand numbers and
can't dial.''

`I always start by stitching the upper portion of the jutti from the tip
of the sole. The person who manages to do this right is a craftsman,
others are not', he says

Hans Raj has never been to school, but finds great joy in exploring new
places beyond his village, especially with his friend, a holy man in the
neighbouring village: ``Every year we take a trip. He has his own car,
and he often invites me to join him on travels. Together, with one or
two more people, we've visited places in Haryana and Alwar and Bikaner
in Rajasthan.''

\begin{center}\rule{0.5\linewidth}{0.5pt}\end{center}

It is well past 4 p.m, and Rupana village is bathed in the warm glow of
a lingering mid-November sun. One of Hans Raj's loyal customers has
arrived with a friend to pick up a pair of Punjabi juttis . ``Could you
also make a jutti for him by tomorrow?'' he asks Hans Raj. The friend
has come from far away -- Tohana in Haryana -- 175 kilometres from here.

Hans Raj smiles, responding to the customer's request with a friendly,
`` yaar , it is not possible by tomorrow.'' The customer, however, is
persistent: ``Muktsar is renowned for Punjabi juttis .'' The customer
then turns to us saying, ``there are thousands of jutti shops in the
city. But here in Rupana, it is only he who crafts them by hand. We are
familiar with his work.''

The customer tells us that till Diwali, the entire shop was filled with
juttis . A month later in November, only 14 pairs remain. What makes
Hans Raj's juttis so special? ``The ones he makes are flatter in the
middle,'' the customer says, pointing to the juttis hanging on the wall,
``The difference lies in the hands {[}of the craftsperson{]}.

`There are thousands of jutti shops in the city. But here in Rupana, it
is only he who crafts them by hand,' says one of Hans Raj's customers

Hans Raj doesn't work alone -- he gets some of the juttis stitched by
Sant Ram, a skilled shoemaker in his native village, Khunan Khurd, 12
kms away. During Diwali or the paddy season, when the demand is high, he
outsources his work, paying Rs. 80 for stitching a pair.

The master shoemaker tells us the difference between a craftsman and a
workman: ``I always start by stitching the panna {[}upper portion{]} of
the jutti from the tip of the sole. This is the most challenging phase
of crafting juttis . The person who manages to do this right is a
mistiri {[}craftsman{]}, others are not.''

It wasn't a skill he learnt easily. ``Initially, I was not good at
stitching shoes with thread,'' Hans Raj recalls. ``But when I committed
to learning it, I mastered it in two months. The rest of the skill I
picked up over time, first by asking my father, and later by observing
him,'' he adds.

Over the years, he has innovated, incorporating a technique of stitching
small strips of leather on both sides of the jutti , seamlessly
connecting all the joints. ``These small strips add strength to the
jutti . The shoes become more resistant to breakage,'' he explains.

The craft of jutti- making requires precision. `Initially, I was not
good at stitching shoes with thread,' he recalls. But once he put his
mind to it, he learnt it in two months

Hans Raj and his family of four, including his wife, Veerpal Kaur, and
two sons and a daughter -- now married and parents themselves --
relocated from Khunan Khurd to Rupana some 18 years ago. At that time,
their eldest son, who is 36 now, began working at the paper mill in the
village here.

``It was mostly {[}Dalit{]} families who made juttis in Khunan Khurd,
working from their homes. As time passed, the new generation didn't
learn the craft. And those who knew, passed away,'' says Hans Raj.

Today, in his old village, only three craftsmen, all from his community
of Ramdasi Chamars (listed as Scheduled Caste in the state), are still
engaged in the art of handcrafting Punjabi juttis while Hans Raj is the
only one in Rupana.

``We saw no future for our children in Khunan Khurd, so we sold our
property there and bought one here,'' Veerpal Kaur says, her voice a mix
of determination and hope. She speaks fluent Hindi, a result of the
diversity in the neighbourhood which is populated by migrants from Uttar
Pradesh and Bihar, many of whom work in the paper mill and live in
rented rooms in the vicinity.

Veerpal Kaur, Hans Raj's wife, learnt to embroider juttis from her
mother-in-law. She prefers to sit alone while she works, without any
distractions

It takes her about an hour to embroider one pair. She uses sharp needles
that can pierce her fingers if she is not careful, Veerpal says

This is not the first time Hans Raj's family has migrated. ``My father
came to Punjab from Narnaul {[}in Haryana{]} and started making juttis
,'' says Hans Raj.

A 2017 study by the Guru Nanak College of Girls in Sri Muktsar Sahib
district shows that thousands of jutti -makers families migrated from
Rajasthan to Punjab in the 1950s. Hans Raj's ancestral village Narnaul
is situated on the border of Haryana and Rajasthan.

\begin{center}\rule{0.5\linewidth}{0.5pt}\end{center}

``When I started, a pair would only cost Rs. 30. Now a full embroidered
jutti can cost over Rs 2,500,'' Hans Raj recounts.

From the small and large scattered pieces of leather at his workshop,
Hans Raj shows us two kinds: cowhide and buffalo hide. ``The buffalo
hide is used for the sole, and the cowhide is for the upper half of the
shoes,'' he explains, his hands stroking the raw material that once
formed the backbone of the craft.

As he holds up the tanned cowhide he asks if we are comfortable touching
the animal skin. When we express our willingness, he goes on to observe
not just the tanned leather but the contrast. The buffalo hide feels as
thick as 80 paper sheets stacked together. The cowhide, on the other
hand, is much thinner, maybe around 10 paper sheets. In terms of
texture, the buffalo hide has a smoother and stiffer feel, while the
cowhide, though slightly rougher, exhibits greater flexibility and ease
of bending.

Hans Raj opens a stack of thick leather pieces that he uses to make the
soles of the jutti . `Buffalo hide is used for the sole, and the cowhide
is for the upper half of the shoes,' he explains

Left: He soaks the tanned buffalo hide before it can be used. Right: The
upper portion of a jutti made from cow hide

The growing hike in leather prices -- his critical raw material -- and
the switch to shoes and slippers what he calls, `` boot-chappal '' has
led to a fall in the number of people willing to take up this
profession.

Hans Raj treats his tools with great care. For shaping the jutti he uses
a rambi (cutter) to carve and scrape the leather; a morga (wooden
hammer) for beating it until it is stiff and more. The wooden morga
belonged to his father as did a deer horn which he uses to shape the tip
of the shoes from inside as it is difficult to fix only with his hands.

The shoemaker travels to the shoe market in Jalandhar, 170 kms away from
his village, to buy the tanned hides. To reach the mandi (wholesale
market), he takes a bus to Moga and another from Moga to Jalandhar. His
travel costs add up to over Rs. 200 one way.

His most recent journey took place two months prior to Diwali when he
acquired 150 kilograms of tanned leather, valued at Rs. 20,000. Has he
ever faced any trouble carrying the leather, we ask him. ``The concern
is more about transporting untanned leather than the tanned one,'' he
clarifies.

Hans Raj takes great care of all his tools, two of which he has
inherited from his father

he wooden morga {[}hammer{]} he uses to beat the leather with is one of
his inheritances

He visits the mandi to carefully choose the desired quality of leather,
and the traders arrange for its transportation to a nearby city, Muktsar
where he collects it. ``Carrying such heavy material alone on the bus is
anyway not possible,'' he remarks.

Over the years, the material for making juttis has evolved and younger
shoemakers like Raj Kumar and Mahinder Kumar of the Guru Ravidas colony
in Malout say that artificial leather such as rexine and micro cellular
sheets are now more commonly used. Raj and Mahinder, both in their early
forties, belong to the Dalit Jatav community.

``Where a micro sheet cost Rs. 130 per kg, the cowhide now costs ranging
from Rs. 160 to over Rs. 200 per kg,'' says Mahinder. They say leather
has become a rare commodity in the area. ``Earlier, the colony was full
of tanneries and a stench of untanned leather hung in the air. But as
the basti grew, the tanneries were shut down,'' says Raj.

Youngsters are no longer keen on joining the profession, they add, and
low income is not the only reason. ``The stench gets into the clothes,''
Mahinder says, ``and sometimes their friends won't shake their hands.''

oung shoemakers like Raj Kumar (left) say that artificial leather is now
more commonly used for making juttis . In Guru Ravidas Colony in Malout
where he lives and works, tanneries have shut

``In my own family the children don't make juttis ,'' says Hans Raj,
``my sons never entered the shop to understand the craft, how could they
have learned it? Ours is the last generation now to know the skill. I
may also be able to do it for another five years, after me who else will
do it?'' he asks.

As she chops vegetables for dinner, Veerpal Kaur says, ``It isn't
possible to build homes by just making juttis . Almost two years ago,
the family completed the construction of a pucca house, facilitated by
their eldest son's employee loan from the paper mill.

``I had also asked her to learn embroidery, but she didn't learn it
all,'' Hans Raj says, teasing his wife . The two have been married for
38 years. ``I wasn't interested,'' Veerpal chimes back. Based on what
she learnt from her mother-in-law, she can embroider a pair in an hour
at home with zari thread.

Their home, shared with their eldest son's family of three, comprises
two rooms, a kitchen, and a drawing room, with an outdoor toilet.
Adorning the rooms and hall are photos of B.R. Ambedkar and Sant
Ravidas. A similar image of the saint graces Hans Raj's workshop.

ans Raj's juttis have travelled across India with their customers. These
are back in vogue after a gap of about 15 years. Now, `every day feels
like Diwali for me,' a joyous Hans Raj says

``It is in the last 10-15 years that people have started wearing juttis
again,'' Veerpal says,``before that, many had also stopped asking for
the shoemakers.''

During that time, Hans Raj worked as a farm labourer and occasionally
crafted juttis within a day or two when a customer came by.

``Now, more college going boys and girls are interested in wearing these
juttis ,'' Veerpal says.

Customers have also carried juttis to various places, including
Ludhiana, Rajasthan, Gujarat, and Uttar Pradesh. Hans Raj fondly recalls
crafting eight pairs of Punjabi juttis for a mill worker during his last
big order. The mill worker purchased them for his relatives in Uttar
Pradesh.

Since there's a consistent demand for his craftsmanship in his current
location, ``Every day feels like Diwali for me,'' a joyous Hans Raj
says.

In November 2023, a few weeks after this story was reported, Hans Raj
suffered a partial stroke. He is now recovering.

This story is supported by a fellowship from Mrinalini Mukherjee
Foundation (MMF).

\#leather \#dalit \#leather-worker \#dalit-communities \#craftpersons

Sanskriti Talwar

Sanskriti Talwar is an independent journalist based in New Delhi, and a
PARI MMF Fellow for 2023.

Naveen Macro

Naveen Macro is a Delhi-based independent photojournalist and
documentary filmmaker and a PARI MMF Fellow for 2023.

Editor : Sarbajaya Bhattacharya

Sarbajaya Bhattacharya is a Senior Assistant Editor at PARI. She is an
experienced Bangla translator. Based in Kolkata, she is interested in
the history of the city and travel literature.




\end{document}
